\documentclass[../Immersed_Boundary_Method.tex]{subfiles}

\begin{document}

\section{Penalty Immersed Boundary Method (PIB)}

The Penalty Immersed Boundary Methods is a simple approach try to solve fluid-structure interaction with material bounadary that has mass. 

\subsection{Rigid Body pIB}
One kind of problem pIB can simulates is a rigid body in flow. The key of pIB is to consider two sets of Material Points, denoted as $X(q,s,t)$ and $Y(q,s,t)$. Here $q,r$ are Lagrangian Coordinate and $t$ is time. $\bfX$ is the true boundary points that interact masslessly with the flow, and $\bfY$ is called the \textbf{Ghost Mass}. It doesn't interact with the flow but only with the Material points $\bfX$ through an imaginary spring. 
\begin{explanation}
    \begin{enumerate}
        \item At the start, the Material points' position is updated using fluid velocity interpolated at $\bfX$. 
        \begin{lstlisting}
                X_half_top = X_tube_top + dt / 2 * interpolation(u, X_tube_top, Num_b, Nx, Ny, dx, dy);
        \end{lstlisting}
        \item The force acting to the fluid by the Material is modeled by a spring connecting the Ghost point and material point. 
        \begin{equation}
            \mathbf{F}(q, r,t)=K(\mathbf{Y}(q, r,t)-\mathbf{X}(q, r, t))
        \end{equation}
        \item Then spread the force $F$ to Eucleadian space. 
        \item Numerically solve the time evolution of fluid using Navier Stokes solver. 
        \item Update the Material point position $\bfX$. 
    \end{enumerate}
\end{explanation}

\subsubsection{Implementation : Circle in Flow}
\end{document}